\section{Exercise 3: Reduction to MCFP}

\subsection{Exercise 3.1}

Given an instance $I_0$ of GMCFP, where some or
all of the edges $e$ have $l_e = -\infty, u_e = \infty$,
we can create a graph wherein all edges $e'$
have either $l_{e'}$ or $u_{e'}$ finite.

For each such $e$ in $I_0$, in $I_1$ we instead have antiparalell edge pairs $e_+, e_-$
where $l_{e_+} = l_{e_-} = 0, u_{e_+} = u_{e_-} = \infty$. Let $-c_{e_-} = c_{e_+} = c_e$.

For all other edges, and all vertices in $I_0$ an identical counterpart exists in $I_1$.

The property of a negative lower bound is to natively permit a ``reverse'' flow. By
creating an antiparalell edge with an equal negative cost, we model this property
since in $I_0$ if $e$ has flow $x_e > 0$ and therefore cost $c_e | x_e |$, then in $I_1$ we let
$x_{e_+} = | x_e |, x_{e_-} = 0$ and total cost $ c_{e_+} x_{e_+} + c_{e_-} x_{e_-} = c_e | x_e |$; if
$e$ has flow $x_e < 0$ and therefore cost $- c_e | x_e |$, then in $I_1$ we let
$x_{e_-} = | x_e |, x_{e_+} = 0$ and total cost $ c_{e_+} x_{e_+} + c_{e_-} x_{e_-} = - c_e | x_e |$.

A solution to $I_1$ is therefore also a solution to $I_0$, and $I_1$ has at
most twice as many edges as $I_0$

\subsection{Exercise 3.2}

Given an instance $I_1$ of GMFP, where some or all of the edges $e$ have
$l_e = -\infty$ we can create a graph wherein all edges $e'$
have $l_{e'}$ finite.

(By the transformation above, we know that there are no edges in $I_1$ with $l_e = -\infty$
and also $u_e = \infty$.)

For each such $e$ in $I_1$, in $I_2$ we instead have an opposite edge $e_{\rightarrow}$
with $l_{e_{\rightarrow}} = -u_{e}, u_{e_{\rightarrow}} = -l_{e} = \infty, c_{e_{\rightarrow}} = -c_e$.

For all other edges, and all vertices in $I_1$ an identical counterpart exists in $I_2$.

By similar argument to above, a flow $x_e < 0$ becomes a flow $-x_{e_{\rightarrow}} > 0$ in the
opposite direction, with negative and cost of equal magnitude, which is equivalent.

A solution to to $I_2$ is therefore also a solution to $I_1$, and $I_2$ has the same
amount of edges as $I_1$.

\subsection{Exercise 3.3}

Given an instance of GMCFP, $I_2$ where some or all edges $e$ have non-zero, finite $l_e$, we can
create a graph $I_3$ where all edges $e'$ have $l_{e'} = 0$.

(By the transformation above, we know that there are no edges in $I_2$ with $l_e = -\infty$.)

For each such $e$ in $I_2$, in $I_3$ we instead have an edge $e_0$ where
$u_{e_0} = u_e - l_e, l_{e_0} = l_e - l_e = 0$. Additionally, the vertices $(v,u)$ that $e$ connect $e = (v, u)$, 
are replaced by $e_0 = (v_+, u_-)$ where the demand of the two vertices are adjusted as follows:
$b_{v_+} = b_v + l_e, b_{u_-} = b_u - l_e$.

(If multiple vertices with non-zero, finite $l_e$ exit or enter the same vertex, this process is
iterated for each in the obvious manner.)

This adjusts all flows by $x_{e'} = x_{e} - l_{e}$.

For all other edges, and all vertices in $I_2$ an identical counterpart exists in $I_3$.

We keep the costs on each edge, and observe that monotonicity of cost is preserved, and therefore
the minimum. The cost for a given flow in $I_3$ is

\begin{align*}
  \sum_e' c_{e'} x_{e'} &= \sum_e c_{e} ( x_{e} - l_{e} ) \\
  &= \sum_e c_{e} x_{e} - \sum_e c_{e} l_{e} \\
  &= \sum_e c_e x_e - C
\end{align*}

and is the cost of a given flow in $I_2$ plus or minus some constant.

A solution to $I_3$ can therefore be transformed into a solution in $I_2$ by
adding $l_e$ to each flow $x_{e'}$, and $I_3$ has the same amount of edges as $I_2$.

\subsection{Exercise 3.4}

Each of the transformations from $I_1$ to $I_2$ to $I_3$ preserve the number of vertices and
edges, while the one from $I_0$ to $I_1$ potentially doubles the number of edges. This is 
well within $O(|V| + |E|)$.
