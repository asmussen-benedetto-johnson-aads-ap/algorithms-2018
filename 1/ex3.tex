\section{Exercise 3: Reduction to MCFP}

\subsection{Exercise 3.1}

Given an instance $I_0$ of GMCFP, where some or
all of the edges $e$ have $l_e = -\infty, u_e = \infty$, we can
simply apply a ``modified'' version of the exponential function to
all numerical values in $I_0$:
\begin{align}
  \brm{exp}'(x) = \begin{cases}
    e^x & \text{if $x$ is finite} \\
    \lim_{y\to x} e^y & \text{otherwise.}
  \end{cases}
\end{align}
This function has an obvious inverse, constructible using the natural logarithm.

The exponential function is an isomorphic map between the abelian groups of addition
over real numbers, and multiplication over strictly positive real numbers --- the inverse
is of course, the natural lograrithm.
\begin{align}
  \brm{log}'(x) = \begin{cases}
    \mathrm{ln} x & \text{if $x$ is finite} \\
    \lim_{y\to x} \mathrm{ln} y & \text{otherwise.}
  \end{cases}
\end{align}

The exponential function and natural logarithm are also monotonic, meaning they
preserves minimum, maximum, and all inequalities; and assuming strictly positive costs, multiplying
flow by cost is also a monotonic operation, meaning any solution to $I_1$ is also a solution to $I_0$.

\subsection{Exercise 3.2}

$I_2 = I_1$, see above, and not that every $l_e$ is not just finite, but also positive or zero, as
we simply map the domain of $l_e$ through the modified exponential to find the range:
$\brm{exp}'(\{-\infty\}\cup \mathbb R) = [0;\infty).$

\subsection{Exercise 3.3}

Given an instance of GMCFP, $I_2$ where all $l_e$ are positive or zero, we can
adjust the values of $l_e$, $u_e$, $f_e$ and $d$ with addition --- which is
isomorphic and monotonic, and therefore solution-preserving.

First, we simply subtract $l_e$ from each $l_e$, $u_e$, and $f_e$ in $I_3$, giving us the
desired property that $l_e = 0$.

Second, we run a maximum flow solver on the graph in $I_2$, using $l_e$ as the capacity. This
gives us the number $d'$ needed to subtract from the demand in $I_3$ in order to ensure we are still
solving for the same demand.

\subsection{Exercise 3.4}

Each step has exactly the same number of edges as we are only modifying the numerical values; at
no point do we add or remove edges or vertices.
