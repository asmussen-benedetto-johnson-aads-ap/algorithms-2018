\section{Exercise 3: Reduction to MCFP}

\subsection{Exercise 3.1}

Given an instance $I_0$ of GMCFP, where some or
all of the edges $e$ have $l_e = -\infty, u_e = \infty$, we can
simply apply a ``modified'' version of the exponential function to
all numerical values in $I_0$:
\begin{align}
  \brm{exp}(x) = \begin{cases}
    e^x & \text{if $x$ is finite} \\
    \lim_{y\to x} e^y & \text{otherwise.}
  \end{cases}
\end{align}
This function has an obvious inverse, constructible using the natural logarithm.

The exponential function is an isomorphic map between the abelian groups of addition
over real numbers, and multiplication over strictly positive real numbers.

\subsection{Exercise 3.2}

See above.

\subsection{Exercise 3.3}

See above.

\subsection{Exercise 3.4}

It has exactly the same number of edges as we are only modifying the numerical values.
